\section{Professional experience}\label{professional-experience}

\begin{description}
\itemsep3pt\parskip0pt\parsep0pt
\item[2016-]
Interdisciplinary Postdoctoral Scientist The Field Museum of Natural
History, Chicago IL Division of Birds Advisor: Shannon Hackett
\item[2014-2016]
Postdoctoral Scholar, Jackson School of Geosciences\\The University of
Texas at Austin, Austin TX\\Advisor: Julia Clarke
\end{description}

\section{Education}\label{education}

\begin{description}
\itemsep3pt\parskip0pt\parsep0pt
\item[2014]
Ph.D.\ Integrated Bioscience The University of Akron, Akron OH
Dissertation: ``Mechanisms and evolution of iridescent feather colors in
birds'' Advisor: Matthew D. Shawkey
\item[2006]
M.S. Education The University of Akron, Akron OH\}
\hfill 2006\textbackslash{}
\item[2002]
B.S. Biology Baldwin-Wallace University, Berea OH
\end{description}

\section{Publications}\label{publications}

\begin{description}
\itemsep3pt\parskip0pt\parsep0pt
\item[2016]
Riede T, \textbf{Eliason CM}, Miller EH, Goller F, Clarke JA. 2016.
Coos, booms, and hoots: the evolution of closed-mouth vocal behavior in
birds. \journal{Evolution} 70:1734-1746.
\highlight{(Received extensive media coverage, including The Tonight Night Show with Jimmy Fallon, Time Magazine, NPR Weekend Edition)}
\item[2016]
Iskandar J-P*, \textbf{Eliason CM}, Astrop T, Igic B, Maia R, Shawkey
MD. 2016. Morphological basis of glossy red plumage colors.
\journal{Biological Journal of the Linnaean Society} 119:477-487.
\item[2016]
\textbf{Eliason CM}, Shawkey MD, Clarke JA. 2016. Evolutionary shifts in
the melanin-based color system of birds. \journal{Evolution} 70:445-455.
\item[2015]
\textbf{Eliason CM}, Maia R, Shawkey MD. 2015. Modular color evolution
in ducks facilitated by a complex nanostructure. \journal{Evolution}
69:357-367.
\item[2014]
\textbf{Eliason CM}, Shawkey MD. 2014. Antireflection-enhanced color by
a natural graded refracting index (GRIN) structure.
\journal{Optics Express} 22:A642-A650.
\highlight{(Highlighted in Virtual Journal of Biomedical Optics)}
\item[2014]
D'Alba LD, Jones DN, \textbf{Eliason CM}, Badawy HT, Shawkey MD. 2014.
Antimicrobial properties of a nanostructured eggshell from a
compost-nesting bird. \journal{Journal of Experimental Biology}
217:116-1121.
\item[2013]
\textbf{Eliason CM}, Bitton, P-P, Shawkey MD. 2013. How hollow
melanosomes affect iridescent colour production in birds.
\journal{Proceedings of the Royal Society: B} 280:20131505.
\item[2013]
Maia R, \textbf{Eliason CM}, Bitton, P-P, Doucet SM, Shawkey MD. 2013.
pavo: an R package for the analysis, visualization and organization of
spectral data. \journal{Methods in Ecology and Evolution} 4:906-913.
\item[2012]
\textbf{Eliason CM}, Shawkey MD. 2012. A photonic heterostructure
produces diverse iridescent colours in duck wing patches.
\journal{Journal of the Royal Society Interface} 9(74):2279-2289.
\highlight{(Received press coverage in Science and Spiegel Online)}
\item[2011]
\textbf{Eliason CM}, Shawkey MD. 2011. Decreased hydrophobicity of
iridescent feathers: a potential cost of shiny plumage.
\journal{Journal of Experimental Biology} 214:2157-2163.
\highlight{(Named as Editor's Choice for that issue of JEB, and as one of the top eight articles of the year; received press coverage in Spiegel Online)}
\item[2011]
Shawkey MD, D'Alba L, Wozny J, \textbf{Eliason CM}, Koop JAH, Jia L.
2011. Structural color change following hydration and dehydration of
iridescent mourning dove (Zenaida macroura) feathers.
\journal{Zoology (Jena)} 114:59-68.
\item[2010]
\textbf{Eliason CM}, Shawkey MD. 2010. Rapid, reversible response of
iridescent feather color to ambient humidity. \journal{Optics Express}
18:21284-92.
\item[2007]
Blackledge TA, \textbf{Eliason CM} 2007. Functionally independent
components of prey capture are architecturally constrained in spider orb
webs. \journal{Biology Letters} 3:456-458.
(\href{http://www.nature.com/articles/srep25936.pdf}{PDF})
\end{description}

\section{Awards and scholarships}\label{awards-and-scholarships}

\begin{description}
\itemsep3pt\parskip0pt\parsep0pt
\item[2015]
David H. Smith Conservation Postdoctoral Research Fellowship
(\href{http://conbio.org/mini-sites/smith-fellows/meet-the-fellows/2015-fellows/sean-anderson}{URL})
\item[2015]
Simon Fraser University Dean of Graduate Studies Convocation Medal
(\href{https://www.sfu.ca/dean-gradstudies/blog/year/2015/06/SeanAnderson.html}{URL})
\item[2014]
Garfield Weston Foundation / BC Packers Ltd. Graduate Fellowship in
Marine Sciences
\item[2014]
Graduate Fellowship (two semesters), Simon Fraser University
\item[2012--13]
Canadian Fulbright Scholar award to the University of Washington
\item[2011]
Canadian \href{http://goo.gl/nA1zE}{Governor General's Academic Gold
Medal} for the top-ranked Master's Natural Sciences and Engineering
thesis at Dalhousie University in 2010
\item[2011--14]
Provost Prize of Distinction, Simon Fraser University
\item[2011--14]
Natural Sciences and Engineering Research Council of Canada Postgraduate
Scholarship (Doctoral)
\item[2007--10]
Faculty Research Grant Scholarship, Dalhousie University
\item[2007--09]
Graduate Studies Scholarship, Dalhousie University
\item[2007]
Environmental Programmes Honour Society Medal, Dalhousie University
\end{description}

\section{Software}\label{software}

\begin{description}
\itemsep3pt\parskip0pt\parsep0pt
\item[2014]
\textbf{Anderson, S.C.}, J.W. Moore, M.M McClure, N.K. Dulvy, A.B.
Cooper.\\ metafolio: Metapopulation simulations for conserving salmon
through portfolio optimization.
\url{http://cran.r-project.org/package=metafolio}
\item[2013]
\textbf{Anderson, S.C.}, C.C. Monnahan, K.F. Johnson, K. Ono, J.L.
Valero, C.J Cunningham, F. Hurtado-Ferro, R. Licandeo, C.R. McGilliard,
C.S. Szuwalski, K.A. Vert-pre, A.R. Whitten. ss3sim: Fisheries stock
assessment simulation testing with Stock Synthesis.
\url{http://cran.r-project.org/package=ss3sim}
\item[2013]
\textbf{Anderson, S.C.}, A.B. Cooper, N.K. Dulvy. ecofolio: Tools to
quantify metapopulation portfolio effects.
\url{https://github.com/seananderson/ecofolio}
\end{description}

\section{Invited talks and conference
presentations}\label{invited-talks-and-conference-presentations}

\begin{description}
\itemsep3pt\parskip0pt\parsep0pt
\item[2016]
\textbf{Anderson, S.C.}, A.B. Cooper, O.P. Jensen, C. Minto, J.T.
Thorson, J.C. Walsh, M. Dickey-Collas, K.M. Kleisner, C. Longo, G.C.
Osio, D. Ovando, I. Mosqueira, A.A. Rosenberg, E.R. Selig. Improving
estimates of population status and trend with superensemble models.
World Fisheries Conference, Busan, South Korea. (Presented \emph{in
absentia} by J.T. Thorson.)
\item[2015]
\textbf{Anderson, S.C.}, A.B. Cooper, O.P. Jensen, C. Minto, J.T.
Thorson, J.C. Walsh, M. Dickey-Collas, K.M. Kleisner, C. Longo, G.C.
Osio, D. Ovando, I. Mosqueira, A.A. Rosenberg, E.R. Selig. Improving
estimates of population status and trend with superensemble models.
American Fisheries Society Annual Meeting, Portland, OR, United States.
\item[2013]
\textbf{Anderson, S.C.}, A.B. Cooper, N.K. Dulvy. False prophets: The
challenges of quantifying ecological portfolios
(\href{http://seananderson.ca/talks/2013/PE_SAFS_quantsem.pdf}{slides}),
Quantitative Seminar, School of Aquatic and Fishery Sciences, University
of Washington, Seattle, WA, United States.
\item[2012]
\textbf{Anderson, S.C.}, A.B. Cooper, N.K. Dulvy. False prophets: The
ecological portfolio effect overestimates the benefit of diversity.
Branch Lab, School of Aquatic and Fishery Sciences, University of
Washington, Seattle, WA, United States.
\item[2012]
\textbf{Anderson, S.C.}, A.B. Cooper, N.K. Dulvy. Metapopulation
dynamics and the generality of the ecological portfolio effect. Les
Ecologistes Departmental Seminar, Simon Fraser University, Burnaby, BC,
Canada.
\item[2010]
\textbf{Anderson, S.C.} The rise of invertebrates, the fall of sea
cucumbers, and the risk of maturing late. Earth2Ocean Research Group,
Simon Fraser University, Burnaby, BC, Canada.
\item[2009]
\textbf{Anderson, S.C.}, J.E. Mills Flemming, R. Watson, H.K. Lotze.
Ecosystem impacts of the global expansion of invertebrate fisheries.
International Oceans Past II Conference \emph{Multidisciplinary
Perspectives on the History and Future of Marine Animal Populations}.
Vancouver, BC, Canada.
\item[2009]
\textbf{Anderson, S.C.}, J.E Mills Flemming, R. Watson, H.K. Lotze.
Global invertebrate fisheries: trends and consequences. NCEAS (National
Center for Ecological Analysis and Synthesis) Working Group
\emph{Finding Common Ground in Marine Conservation and Management}.
Santa Barbara, CA, USA.
\item[2009]
\textbf{Anderson, S.C.}, H.K. Lotze, N.L. Shackell. Evaluating the
knowledge base for expanding low-trophic level fisheries in Atlantic
Canada. Harvest Fisheries Seminar Series. Bedford Institute of
Oceanography, Fisheries and Oceans, Dartmouth, NS, Canada.
\item[2008]
\textbf{Anderson, S.C.}, H.K. Lotze, N.L. Shackell. Evaluating the
knowledge base for expanding low-trophic level fisheries in Atlantic
Canada. Departmental Seminar, Biology Department, Dalhousie University,
Canada.
\end{description}

\section{Teaching}\label{teaching}

\begin{description}
\itemsep3pt\parskip0pt\parsep0pt
\item[2014--16]
Developed self-directed lecture and exercises on ggplot2 for FISH 554:
Beautiful graphics in R, School of Aquatic and Fishery Sciences,
University of Washington, Seattle, WA, United States.\\
\url{http://seananderson.ca/ggplot2-FISH554/}
\item[2013--14]
Organizer of Stats Beerz --- a statistical help group attended by
graduate students and postdocs primarily in the Earth to Oceans research
group, but also the wider SFU Biology and Geography Departments, and the
School of Resource and Environmental Management (REM).
\item[2013]
Two-part workshop on data manipulation for Stats Beerz and Earth to
Oceans groups at Simon Fraser University with approximately 25
participants. An introduction to plyr, advanced concepts with plyr and
function debugging, and an introduction to dplyr.\\
\url{https://github.com/seananderson/plyr-statsbeerz}
\item[2013]
Instructor for BISC-888-1: Data Wrangling and Visualization in R, a
graduate-level course at Simon Fraser University, Burnaby, BC, Canada
with approximately 20 participants. Co-developed curriculum and
developed/delivered lectures, exercises, notes, and assignments for
three of six two-hour classes.\\
\url{https://github.com/seananderson/datawranglR} (see classes 03, 04,
05)
\item[2012]
Introduction to ggplot2.
(\href{http://seananderson.ca/courses/12-ggplot2/ggplot2_notes.pdf}{notes},
\href{http://seananderson.ca/courses/12-ggplot2/ggplot2_slides_with_examples.pdf}{slides})
Lecture for FISH 507H: Beautiful Graphics in R, School of Aquatic and
Fishery Sciences, University of Washington, Seattle, WA, United States
\item[2012]
Workshop on the R package plyr.
(\href{http://seananderson.ca/courses/12-plyr/plyr_2012.pdf}{notes},
\href{http://seananderson.ca/courses/12-plyr/plyr_2012_slides.pdf}{slides},
\href{http://seananderson.ca/courses/12-plyr/plyr_2012_examples.html}{examples})
Branch Lab, School of Aquatic and Fishery Sciences, University of
Washington, Seattle, WA, United States
\item[2011]
Multipanel plotting in R with base graphics.
(\href{http://seananderson.ca/courses/11-multipanel/multipanel.pdf}{notes},
\href{http://seananderson.ca/courses/11-multipanel/multipanel-slides.pdf}{slides}).
Earth2Ocean Research Group, Simon Fraser University, Burnaby, BC,
Canada.
\item[2011]
A brief introduction to R.
(\href{http://seananderson.ca/courses/11-rintro/RIntro.pdf}{notes},
\href{http://seananderson.ca/courses/11-rintro/RIntro.R}{workshop
code}). Earth2Ocean Research Group, Simon Fraser University, Burnaby,
BC, Canada.
\item[2011]
Workshop on the R package plyr
(\href{http://seananderson.ca/courses/11-plyr/plyr.pdf}{notes},
\href{http://seananderson.ca/courses/11-plyr/plyr-slides.pdf}{slides}).
Earth2Ocean Research Group, Simon Fraser University, Burnaby, BC,
Canada.
\item[2007--08]
Teaching Assistant, Organismal Biology and Ecology, Dalhousie
University, two semesters (BIOL 1021).
\item[2007--08]
Teaching Assistant, Marine Mammology, Dalhousie University (BIOL 4060).
\end{description}

\section{Working groups and
workshops}\label{working-groups-and-workshops}

\begin{description}
\itemsep3pt\parskip0pt\parsep0pt
\item[2015-16]
NCEAS (National Center for Ecological Analysis and Synthesis, Santa
Barbara, CA) \emph{Applying portfolio effects to the Gulf of Alaska
ecosystem: Did multi-scale diversity buffer against the Exxon Valdez oil
spill?} (\href{https://www.nceas.ucsb.edu/featured/marshall}{URL})
\item[2015-16]
Gordon and Betty Moore Foundation funded working group \emph{Applying
data-limited stock status models and developing management guidance for
unassessed fish stocks}.
\item[2011--13]
NESCent (National Evolutionary Synthesis Center, Durham, NC) Working
Group \emph{Determinants of Extinction in Ancient and Modern Seas} led
by Paul Harnik, Seth Finnegan, and Rowan Lockwood.
(\href{http://www.nescent.org/science/awards_summary.php?id=256}{URL})
\item[2010]
Atlantic Halibut Assessment Science Peer Review Meeting, Fisheries and
Oceans, Dartmouth, NS, Canada.
\item[2007--09]
NCEAS (National Center for Ecological Analysis and Synthesis, Santa
Barbara, CA) Distributed Graduate Seminar, in association with the
Working Group \emph{Finding Common Ground in Marine Conservation and
Management} led by Ray Hilborn and Boris Worm.
(\href{http://www.nceas.ucsb.edu/projects/12307}{URL})
\item[2008]
Workshop on Canadian Science and Management Strategies for Sea Cucumber
(\emph{Cucumaria frondosa}), Fisheries and Oceans, Dartmouth, NS, Canada
(work presented \emph{in absentia}).
\item[2007]
Workshop on Canadian Science and Management Strategies for Atlantic
Hagfish, Fisheries and Oceans, Dartmouth, NS, Canada.
(\href{http://www.dfo-mpo.gc.ca/CSAS/Csas/Publications/Pro-CR/2009/2009_009_e.htm}{URL})
\end{description}

\section{Reviews}\label{reviews}

Reviewer for Science, Ecology, Ecological Applications, Conservation
Biology, Fish and Fisheries, Marine Policy, Population Ecology,
International Journal of Tropical Biology and Conservation, Journal of
Environmental Management, Endangered Species Research, Aquatic
Conservation

0
